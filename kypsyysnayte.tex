
% --- Template for thesis / report with tktltiki2 class ---
%
% last updated 2013/02/15 for tkltiki2 v1.02

\documentclass[finnish]{tktltiki2}

  % tktltiki2 automatically loads babel, so you can simply
  % give the language parameter (e.g. finnish, swedish, english, british) as
  % a parameter for the class: \documentclass[finnish]{tktltiki2}.
  % The information on title and abstract is generated automatically depending on
  % the language, see below if you need to change any of these manually.
  %
  % Class options:
  % - grading                 -- Print labels for grading information on the front page.
  % - disablelastpagecounter  -- Disables the automatic generation of page number information
  %                              in the abstract. See also \numberofpagesinformation{} command below.
  %
  % The class also respects the following options of article class:
  %   10pt, 11pt, 12pt, final, draft, oneside, twoside,
  %   openright, openany, onecolumn, twocolumn, leqno, fleqn
  %
  % The default font size is 11pt. The paper size used is A4, other sizes are not supported.
  %
  % rubber: module pdftex

  % --- General packages ---

  \usepackage[utf8]{inputenc}
  \usepackage[T1]{fontenc}
  \usepackage{lmodern}
  \usepackage{microtype}
  \usepackage{amsfonts,amsmath,amssymb,amsthm,booktabs,color,enumitem,graphicx}
  \usepackage[pdftex,hidelinks]{hyperref}

  \graphicspath{{images/}}

  % Automaticall set the PDF metadata fields
  \makeatletter
  \AtBeginDocument{\hypersetup{pdftitle = {\@title}, pdfauthor = {\@author}}}
  \makeatother

  % --- Language-related settings ---
  %
  % these should be modified according to your language

  % babelbib for non-english bibliography using bibtex
  \usepackage[fixlanguage]{babelbib}
  \selectbiblanguage{finnish}

  % add bibliography to the table of contents
  % --- tktltiki2 options ---
  %
  % The following commands define the information used to generate title and
  % abstract pages. The following entries should be always specified:

  \title{Neuroverkkojen peruskäsitteet}
  \author{Teemu Sarapisto}
  \date{\today}
  \level{Kypsyysnäyte}
  % The following can be used to specify keywords and classification of the paper:

  % classification according to ACM Computing Classification System (http://www.acm.org/about/class/)
  % This is probably mostly relevant for computer scientists
  % uncomment the following; contents of \classification will be printed under the abstract with a title
  % "ACM Computing Classification System (CCS):"
  % \classification{}

  % If the automatic page number counting is not working as desired in your case,
  % uncomment the following to manually set the number of pages displayed in the abstract page:
  %
  % \numberofpagesinformation{16 sivua + 10 sivua liitteissä}
  %
  % If you are not a computer scientist, you will want to uncomment the following by hand and specify
  % your department, faculty and subject by hand:
  %
  % \faculty{Matemaattis-luonnontieteellinen}
  % \department{Tietojenkäsittelytieteen laitos}
  % \subject{Tietojenkäsittelytiede}
  %
  % If you are not from the University of Helsinki, then you will most likely want to set these also:
  %
  % \university{Helsingin Yliopisto}
  % \universitylong{HELSINGIN YLIOPISTO --- HELSINGFORS UNIVERSITET --- UNIVERSITY OF HELSINKI} % displayed on the top of the abstract page
  % \city{Helsinki}
  %


  \begin{document}

  % --- Front matter ---


  \maketitle


  % --- Main matter ---

  \mainmatter       % clear page, start arabic page numbering

  Keinotekoisissa neuroverkoissa kulminoituu vuosikymmenten tutkimus funktioiden approksimointiin. Neuroverkot mahdollistavat sellaisten tehtävien automatisoinnin, jotka ovat ennen olleet vain ihmisten tehtävissä. Kaikkivoipia ne eivät kuitenkaan ole, vaan soveltuvat erityisesti ongelmiin, joissa halutaan kuvata jokin syötevektori ulostulovektoriksi. Tälläinen ongelma on esimerkiksi kuvien sisällön tunnistaminen, jossa kuvan pikseliarvot voidaan tulkita olevan syötevektori, ja ulostulovektorin olevan joukko todennäköisyyksiä jossa jokaista tunnistettavaa asiaa kohden neuroverkko antaa todennäköisyysarvion sille, löytyykö kyseinen asia kuvasta.

  Neuroverkot koostuvat kerroksittain toisiinsa yhdistetyistä neuroneista (mainitse että niiden väliset yhteydet sisältävät varsinaisen oppimisen jonka perusteella se sit approksimoi?). Tyypillisesti ensimmäisenä verkossa on niin kutsuttu syötekerros, johon neuroverkon syöte koodataan. Tämän jälkeen seuraa yksi tai useampia piilokerroksia (jotka tekevät mitä?). Viimeisenä kerroksena toimii ulostulokerros, josta neuroverkon laskennan tulos on luettavissa.

  Yleensä neuronit eivät ole saman kerroksen sisällä yhdistettyjä toisiinsa. Tapoja yhdistää neuronit kerroksien välillä toisiinsa löytyy useita, ja monissa neuroverkkototeutuksissa saman verkon sisällä onkin käytety useita eri kerroksien yhdistämismenetelimä
  
Neuronit tuottavat syötteidensä ja sisältämänsä aktivaatiofunktion perusteella jonkin yksittäisen ulostuloarvon, joka taas syötetään verkossa eteenpäin seuraaville neuroneille.
  

  Neuroverkkoa opetettaessa sen onnistumista tehtävässään mitataan yleensä virhefunktiolla, jonka arvo kertoo, kuinka paljon neuroverkon tuottama approksimaatio eroaa tuloksesta joka siltä oltaisiin haluttu. Neuroverkon kouluttamisen voidaan siis ajatella olevan tämän virhefunktion arvon minimoimista harjoitusdataa hyödyntämällä.

  Neuroverkkojen oppimisella tai harjoittamisella tarkoitetaan niiden neuronien välisten painotuksien säätämistä. Tähän on olemassa useita menetelmiä, mutta tällä hetkellä yleisimmin käytössä on niin kutsuttu gradienttimenetelmä.

  Koska neuroverkon toiminta riippuu sen neuroneiden syötteiden painotuksista, myös virhefunktion arvo riippuu niistä. Joten virhefunktion minimoimiseksi 

  Jotta gradienttimenetelmää varten tarvittava virhefunktion gradientti jolla virhefunktion arvoa minimoidaan saadaan muodostettua, tarvitaan verkon painojen osittaisderivaatat  soveltaa virhefunktion minimointiin, tarvitaan verkon painojen osittaisderivaatat suhteessa Takaisinvirtausalgoritmin avulla virhefunktion

  Oikein käytettynä neuroverkot ovat toistaiseksi kehitetyistä ohjatun oppimisen menetelmistä parhaita  löytämään harjoitusmateriaalistaan yleistettävissä olevia piirteitä. Näiden piirteiden avulla on mahdollista esimerkiksi kuvien sisällön tunnistamiseen erikoistuneiden neuroverkkojen kohdalla tunnistaa kuvista joita neuroverkko ei ole ennen nähnyt, löytyykö niistä vaikkapa kissoja.

  Kuten muidenkin koneoppimismenetelmien kanssa, neuroverkkojenkin kanssa törmätään usein ongelmaan,jossa neuroverkko on erikoistunut liikaa harjoitusdataansa ja saa toivotusti virhefunktiosta harjoitusdatalleen pieniä arvoja, mutta harjoitusdatan ulkopuolisten syötteiden kanssa toimii huonosti.
  
  yksi suurimmista haasteista neuroverkkojen käytössä on saada luotava malli oppimaan harjoitusdatastaan piirteitä, jotka yleistyvät myös harjoitusdatan ulkopuolisiin tapauksiin.


  - outro

  - outro

  - outro

  - outro

  - outro

  - outro

  - outro

  - outro





  \end{document}

