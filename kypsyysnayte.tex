
% --- Template for thesis / report with tktltiki2 class ---
%
% last updated 2013/02/15 for tkltiki2 v1.02

\documentclass[finnish]{tktltiki2}

  % tktltiki2 automatically loads babel, so you can simply
  % give the language parameter (e.g. finnish, swedish, english, british) as
  % a parameter for the class: \documentclass[finnish]{tktltiki2}.
  % The information on title and abstract is generated automatically depending on
  % the language, see below if you need to change any of these manually.
  %
  % Class options:
  % - grading                 -- Print labels for grading information on the front page.
  % - disablelastpagecounter  -- Disables the automatic generation of page number information
  %                              in the abstract. See also \numberofpagesinformation{} command below.
  %
  % The class also respects the following options of article class:
  %   10pt, 11pt, 12pt, final, draft, oneside, twoside,
  %   openright, openany, onecolumn, twocolumn, leqno, fleqn
  %
  % The default font size is 11pt. The paper size used is A4, other sizes are not supported.
  %
  % rubber: module pdftex

  % --- General packages ---

  \usepackage[utf8]{inputenc}
  \usepackage[T1]{fontenc}
  \usepackage{lmodern}
  \usepackage{microtype}
  \usepackage{amsfonts,amsmath,amssymb,amsthm,booktabs,color,enumitem,graphicx}
  \usepackage[pdftex,hidelinks]{hyperref}

  \graphicspath{{images/}}

  % Automaticall set the PDF metadata fields
  \makeatletter
  \AtBeginDocument{\hypersetup{pdftitle = {\@title}, pdfauthor = {\@author}}}
  \makeatother

  % --- Language-related settings ---
  %
  % these should be modified according to your language

  % babelbib for non-english bibliography using bibtex
  \usepackage[fixlanguage]{babelbib}
  \selectbiblanguage{finnish}

  % add bibliography to the table of contents
  % --- tktltiki2 options ---
  %
  % The following commands define the information used to generate title and
  % abstract pages. The following entries should be always specified:

  \title{Neuroverkkojen peruskäsitteet}
  \author{Teemu Sarapisto}
  \date{\today}
  \level{Kypsyysnäyte}
  % The following can be used to specify keywords and classification of the paper:

  % classification according to ACM Computing Classification System (http://www.acm.org/about/class/)
  % This is probably mostly relevant for computer scientists
  % uncomment the following; contents of \classification will be printed under the abstract with a title
  % "ACM Computing Classification System (CCS):"
  % \classification{}

  % If the automatic page number counting is not working as desired in your case,
  % uncomment the following to manually set the number of pages displayed in the abstract page:
  %
  % \numberofpagesinformation{16 sivua + 10 sivua liitteissä}
  %
  % If you are not a computer scientist, you will want to uncomment the following by hand and specify
  % your department, faculty and subject by hand:
  %
  % \faculty{Matemaattis-luonnontieteellinen}
  % \department{Tietojenkäsittelytieteen laitos}
  % \subject{Tietojenkäsittelytiede}
  %
  % If you are not from the University of Helsinki, then you will most likely want to set these also:
  %
  % \university{Helsingin Yliopisto}
  % \universitylong{HELSINGIN YLIOPISTO --- HELSINGFORS UNIVERSITET --- UNIVERSITY OF HELSINKI} % displayed on the top of the abstract page
  % \city{Helsinki}
  %


  \begin{document}

  % --- Front matter ---


  \maketitle


  % --- Main matter ---

  \mainmatter       % clear page, start arabic page numbering

  Keinotekoisissa neuroverkoissa kulminoituu vuosikymmenten tutkimus funktioiden approksimoinnin menetelmiin. Neuroverkot ovat viimevuosina mahdollistaneet uusien sellaisten tehtävien automatisoinnin, joita on pidetty vain ihmisille mahdollisina. Kaikkivoipia ne eivät kuitenkaan ole, vaan soveltuvat erityisesti tehtäviin, jotka voidaan tulkita jonkin syötevektorin muuntamiseksi ulostulovektoriksi. Tälläinen tehtävä on esimerkiksi kuvien sisällön tunnistaminen, jossa kuvan pikseleiden väriarvojen voidaan tulkita olevan syötevektori, ja ulostulovektorin olevan jokaista tunnistettavaa asiaa kohden todennäköisyys sille, että kyseinen asia löytyy syötteenä annetusta kuvasta.

  Neuroverkot koostuvat kerroksittain toisiinsa yhdistetyistä laskentayksiköistä joita kutsutaan biologisten esikuviensa mukaisesti neuroneiksi. Tietyssä kerroksessa olevat neuronit saavat syötteenään niitä edeltävän kerroksen neuroneilta ulostuloarvoja ja vastaavasti tämän kerroksen neuroneiden ulostuloarvot toimivat niitä seuraavan kerroksen syötteinä. Neuroniin on tallennettu kullekin syötepoluistaan jokin painotusarvo, joka määrittelee kuinka voimakkaasti kyseistä polkua pitkin tuleva syöte vaikuttaa tämän neuronin ulostuloarvoon. Painotetut syötteet summataan ja syötetään aktivaatiofunktiolle, jonka ulostuloarvo toimii koko neuronin yksittäisenä ulostuloarvona. Aktivaatiofunktiona toimii yleensä jokin helposti derivoitavissa oleva funktio, kuten esimerkiksi usein käytetty sigmoidinen funktio.

   Ensimmäinen neuroverkon kerros on niin kutsuttu syötekerros, johon neuroverkon syöte koodataan. Tämän jälkeen seuraa yksi tai useampia piilokerroksia joissa varsinainen muunnos syötevektorista ulostulovektoriksi tapahtuu. Viimeisenä kerroksena toimii ulostulokerros, josta neuroverkon laskennan tulos on luettavissa. Käytännössä verkkorakenne on kuitenkin vain ajattelua helpottava abstraktio, sillä neuroverkot ovat todellisuudessa tietokonesovelluksissa vain matriiseja joihin verkon kriittiset arvot tallennetaan.

  %(mainitse että niiden väliset yhteydet sisältävät varsinaisen oppimisen jonka perusteella se sit approksimoi?). 
  %Monissa neuroverkkojen käytännön toteutuksissa se, mihin edeltävän kerroksen neuroneihin kukin neuroni on yhdistetty vaihtelee kerroksittain, mutta selkeästi yleisin yhdistämistapa on kuitenkin yhdistää kukin neuroni jokaiseen edeltävän kerroksen neuroniin. Tälläisia kerroksia kutsutaan yleensä täysin yhdistetyiksi kerroksiksi.   

  Neuroverkon rakenteen valitsemisen lisäksi se täytyy harjoittaa jollakin harjoitusaineistolla. Harjoitettaessa neuroverkon tuottaman approksimaation hyvyyttä mitataan yleensä virhefunktiolla. Harjoitusvaiheessa neuroverkolle syötetään jokin harjoitusaineiston yksikkö ja koska toivottu tulos syötteelle tunnetaan, neuroverkon syötteensä perusteella tuottaman tuloksen poikkeama halutusta tuloksesta voidaan mitata virhefunktiolla. Koska neuroverkon tuottamat arvot riippuvat harjoitusvaiheessa enää vain sen neuroneiden syötteiden painotuksista, myös virhefunktion arvo riippuu niistä. Neuroverkon harjoittamisen voidaan siis ajatella olevan tämän virhefunktion arvon minimoimista painotusarvoja säätämällä. Tällä hetkellä yleisimmin käytössä oleva menetelmä virhefunktion minimointiin on niin kutsuttu gradienttimenetelmä, jossa painotuksien osittaisderivaatat virhefunktion suhteen toimivat virhefunktion gradientin komponentteina.

  Painotuksien osittaisderivaattojen selvittämiseen virhefunktion suhteen käytetään takaisinvirtausalgoritmia. Takaisinvirtausalgoritmissa verkon tulkitaan olevan yksi suuri yhdistetty funktio, jolloin virhefunktion derivaatta tietyssä kohtaa verkkoa ja siten tietyn painotuksen kohdalla, on selvitettävissä hyödyntäen derivaatan ketjusääntöä.

  %Oikein sovellettuna neuroverkot ovat toistaiseksi kehitetyistä ohjatun oppimisen menetelmistä parhaita löytämään harjoitusmateriaalistaan yleistettävissä olevia piirteitä. Näiden piirteiden avulla on mahdollista esimerkiksi kuvien sisällön tunnistamiseen erikoistuneiden neuroverkkojen kohdalla tunnistaa kuvista joita neuroverkko ei ole ennen nähnyt, löytyykö niistä vaikkapa kissoja.

  Yksi suurimmista haasteista neuroverkkojen harjoittamisessa on neuroverkon liika erikoistuminen tiettyyn harjoitusaineistoon. Tällöin neuroverkko saa harjoitusaineistosta peräisin oleville syötteille toivotusti virhefunktiosta pieniä arvoja, mutta toimii harjoitusaineiston ulkopuolisten syötteiden kanssa huonosti. Tällöin neuroverkko on epäonnistunut yleistämään harjoitusaineiston piirteitä toivotusti. Ongelman ratkaisuun on kehitetty lukuisia menetelmiä, joista esimerkkinä laskentayksiköiden pudotus, jossa yksittäisiä neuroneita poistetaan satunnaisesti käytöstä estämään tiettyä neuronia erikoistumasta johonkin harjoitusaineiston ominaisuuteen liiaksi.

  Onnistuneesti valituilla verkkorakenteilla ja hyvälaatuisella ja suurella harjoitusaineistolla on mahdollista kehittää neuroverkkoja jotka ovat erittäin hyviä löytämään harjoitusaineistostaan yleistettävissä olevia piirteitä. Jotkin näistä piirteistä ovat usein jopa neuroverkot luoneille ihmisille vaikeita täysin ymmärtää. Neuroverkkojen yllättävän hyvän yleistämiskyvyn ansiosta onkin onnistuttu kehittämään neuroverkkoja hyödyntäviä sovelluksia, jotka suoriutuvat ihmistäkin paremmin tehtävissä joita on pidetty ennen mahdottomina tehdä koneellisesti. Neuroverkot ovat yksi viime vuosien tutkituimmista koneoppimisen menetelmistä, ja uutta kehitystä alueella tapahtuu jatkuvasti. Uusia sovelluskohteita joihin neuroverkot soveltuvat löydetään jatkuvasti, sovelluskohteiden sisältäessä kaikkea sairauksien diagnosoinnista videopelien tekoälyihin.

  \end{document}

