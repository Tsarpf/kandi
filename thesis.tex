% --- Template for thesis / report with tktltiki2 class ---
%
% last updated 2013/02/15 for tkltiki2 v1.02

\documentclass[finnish]{tktltiki2}

  % tktltiki2 automatically loads babel, so you can simply
  % give the language parameter (e.g. finnish, swedish, english, british) as
  % a parameter for the class: \documentclass[finnish]{tktltiki2}.
  % The information on title and abstract is generated automatically depending on
  % the language, see below if you need to change any of these manually.
  %
  % Class options:
  % - grading                 -- Print labels for grading information on the front page.
  % - disablelastpagecounter  -- Disables the automatic generation of page number information
  %                              in the abstract. See also \numberofpagesinformation{} command below.
  %
  % The class also respects the following options of article class:
  %   10pt, 11pt, 12pt, final, draft, oneside, twoside,
  %   openright, openany, onecolumn, twocolumn, leqno, fleqn
  %
  % The default font size is 11pt. The paper size used is A4, other sizes are not supported.
  %
  % rubber: module pdftex

  % --- General packages ---

  \usepackage[utf8]{inputenc}
  \usepackage[T1]{fontenc}
  \usepackage{lmodern}
  \usepackage{microtype}
  \usepackage{amsfonts,amsmath,amssymb,amsthm,booktabs,color,enumitem,graphicx}
  \usepackage[pdftex,hidelinks]{hyperref}

  \graphicspath{{images/}}

  % Automaticall set the PDF metadata fields
  \makeatletter
  \AtBeginDocument{\hypersetup{pdftitle = {\@title}, pdfauthor = {\@author}}}
  \makeatother

  % --- Language-related settings ---
  %
  % these should be modified according to your language

  % babelbib for non-english bibliography using bibtex
  \usepackage[fixlanguage]{babelbib}
  \selectbiblanguage{finnish}

  % add bibliography to the table of contents
  \usepackage[nottoc]{tocbibind}
  % tocbibind renames the bibliography, use the following to change it back
  \settocbibname{Lähteet}

  % --- Theorem environment definitions ---

  \newtheorem{lau}{Lause}
  \newtheorem{lem}[lau]{Lemma}
  \newtheorem{kor}[lau]{Korollaari}

  \theoremstyle{definition}
  \newtheorem{maar}[lau]{Määritelmä}
  \newtheorem{ong}{Ongelma}
  \newtheorem{alg}[lau]{Algoritmi}
  \newtheorem{esim}[lau]{Esimerkki}

  \theoremstyle{remark}
  \newtheorem*{huom}{Huomautus}


  % --- tktltiki2 options ---
  %
  % The following commands define the information used to generate title and
  % abstract pages. The following entries should be always specified:

  \title{Konvolutionaaliset neuroverkot}
  \author{Teemu Sarapisto}
  \date{\today}
  \level{Aine}
  \abstract{TBD}

  % The following can be used to specify keywords and classification of the paper:

  \keywords{avainsana 1, avainsana 2, avainsana 3}

  % classification according to ACM Computing Classification System (http://www.acm.org/about/class/)
  % This is probably mostly relevant for computer scientists
  % uncomment the following; contents of \classification will be printed under the abstract with a title
  % "ACM Computing Classification System (CCS):"
  % \classification{}

  % If the automatic page number counting is not working as desired in your case,
  % uncomment the following to manually set the number of pages displayed in the abstract page:
  %
  % \numberofpagesinformation{16 sivua + 10 sivua liitteissä}
  %
  % If you are not a computer scientist, you will want to uncomment the following by hand and specify
  % your department, faculty and subject by hand:
  %
  % \faculty{Matemaattis-luonnontieteellinen}
  % \department{Tietojenkäsittelytieteen laitos}
  % \subject{Tietojenkäsittelytiede}
  %
  % If you are not from the University of Helsinki, then you will most likely want to set these also:
  %
  % \university{Helsingin Yliopisto}
  % \universitylong{HELSINGIN YLIOPISTO --- HELSINGFORS UNIVERSITET --- UNIVERSITY OF HELSINKI} % displayed on the top of the abstract page
  % \city{Helsinki}
  %


  \begin{document}

  % --- Front matter ---

  \frontmatter      % roman page numbering for front matter

  \maketitle        % title page
  \makeabstract     % abstract page

  \tableofcontents  % table of contents

  % --- Main matter ---

  \mainmatter       % clear page, start arabic page numbering

  \section{Johdanto}

  Viimeisen hieman yli kymmenen vuoden aikana voidaan sanoa keinotekoisten neuroverkkojen ja syväoppimisen tehneen läpimurron. Syväoppimisen voidaan katsoa syntyneen jo 40-luvulla, mutta laajamittaiseen sovelluskäyttöön se on tullut vasta viime vuosina, kun sekä riittävä määrä luokiteltua dataa, että riittävästi prosessointitehoa on tullut helposti saataville. Myös algoritmipuolella tapahtuneet edistykset ovat edesauttaneet läpimurtoa. Aikaisemmin koneoppimisen alalla haasteelliseksi osoittautuneissa sovelluskohteissa kuten kuvien sekä puheen sisällön tunnistamisessa keinotekoiset neuroverkot ovat osoittautuneet toistaiseksi ylivoimaisesti parhaiten toimiviksi ratkaisuiksi.


  TODO: historiaa pidemmälti
  \section{Neuroverkkojen rakenne}
  \subsection{Keinotekoinen neuroni}

    TODO: selkiytä kaavahommelia

    Biologisista vaikuttimistaan huolimatta keinotekoiset neuronit ovat käytännössä Kaavan \ref{eq:neuroni} muotoisia matemaattisia funktioita.

    \begin{equation}
      \label{eq:neuroni}
      \Sigma w_i x_i \mapsto f(\Sigma w_i x_i + b)
    \end{equation}

    Yleisessä muodossaan neuroni ottaa vastaan yhden tai useampia syötteitä $x_1$, $x_2$, ..., $x_n$, joista kullekin on asetettu jokin painoarvo $w_i$. Syötteiden ja painotuksien tulojen summa $\Sigma x_i w_i$ annetaan parametrina aktivaatiofunktiolle $f$ ja tämän funktion arvo toimii neuronin lopullisena ulostuloarvona.

    Toisinaan käytetään myös taipumusvakiota (bias) $b$, joka lisätään syötteiden ja painotuksien tulojen summaan.


    %You can think of the bias as a measure of how easy it is to get the perceptron to output a 1. Or to put it in more biological terms, the bias is a measure of how easy it is to get the perceptron to fire

    Ensimmäinen tällainen neuroni, perseptroni, kehitettiin 50-luvulla. Sen syötteet ja ulostuloarvot ovat binäärisiä ja aktivaatiofunktiona toimii Kaavan \ref{eq:perceptron} mukainen funktio.

    \begin{equation}
      \label{eq:perceptron}
      ulostuloarvo
      \begin{cases}
        0\; jos \; \Sigma x_i w_i + b \leq 0 \\
        1\; jos \; \Sigma x_i w_i + b > 0 \\
      \end{cases}
    \end{equation}

    Yksittäisen neuronin tasolla neuronien oppiminen tapahtuu syötteiden painotuksien ja taipumusarvon muuttumisen kautta. Perseptroneja käytettäessä törmätään kuitenkin usein ongelmaan, jossa yksi pieni muutos painotuksissa tai taipumusarvossa johtaa ulostuloarvon vaihtumiseen, joka saattaa aiheuttaa suuria muutoksia ulostuloarvoissa myös koko neuroverkon tasolla. Usein halutaan hienovaraisempia muutoksia ja tällöin käytetään neuroneita joiden syöte- ja paluuarvot voivat olla myös mitä vain reaalilukuja nollan ja yhden väliltä. Esimerkiksi yksi tällainen laajalti käytössä oleva neuroni on sigmoidinen neuroni, jonka aktivaatiofunktiona toimii sigmoidinen funktio.

  %kappaleessa 3 rojanista hyvää juttua \cite{Rojas96}

  %http://neuralnetworksanddeeplearning.com/chap1.html keksi parempi lähde
  % oliko varmasti 50-luvulla?

  \subsection{Keinotekoisten neuroverkkojen rakenne}

  \begin{figure}[h]
  \label{pic:neuralnet}
  \centering
  \includegraphics[scale=0.5]{basic-neuralnet}
  \caption{http://neuralnetworksanddeeplearning.com/chap1.html}
  \end{figure}

  Yksinkertaisimman verkkorakenteen omaavat eteenpäinsyöttävät neuroverkot muodostetaan tasoittain, jossa jokaisen verkon tason neuronit saavat syötteenään niitä edeltävän tason neuroneiden ulostuloarvot. Poikkeuksena ensimmäinen taso (kuvassa vasemmanpuoleisimpana), joihin verkon syöte koodataan. Esimerkiksi haluttaessa syöttää 64x64 kuva neuroverkolle, voidaan syötekerroksena käyttää 64x64 neuronin kerrosta, johon kuvan pikselien väriarvot koodataan.

  Vaikka syväoppimista voidaan harjoittaa myös muutoin kuin keinotekoisilla neuroverkoilla, neuroverkkojen tapauksessa termillä viitataan neuroverkkojen piilokerroksiin ja niiden määrään. Kasvattamalla neuroverkkotasojen sekä tasoissa olevien neuronien määrää, neuroverkoilla voidaan mallintaa entistä monimutkaisempia funktioita.

  \section{Neuroverkkojen oppiminen}
  Neuroverkkoa opetettaessa tavoitteena on minimoida neuroverkon tekemä virhe sen approksimoidessa jotakin funktiota. Tämän virheen määrää arvioidaan virhefunktion (error function) avulla. Neuroverkon laskiessa ulostuloarvon jollekkin syötteelle, tapahtuu eteenpäinkulkeutumista (forward-propagation). Taaksepäinkulkeutumiseksi (back-propagation) kutsutaan algoritmiä jonka avulla jollekkin neuroverkon painoille määritellylle virhefunktiolle lasketaan gradientti, jonka perusteella neuroverkon approksimoinnin virhettä voidaan lähteä pienentämään gradientin laskeutumismenetelmää käyttäen.

  Suuri haaste neuroverkkojen opetuksessa on ylisovitus (overfitting) jossa neuroverkon virhefunktion arvo on harjoitusdatalla saatu erittäin pieneksi, mutta uuden datan kanssa virhefunktio antaa suuria arvoja. Tällöin neuroverkon oppima malli vastaa harjoitusdataa liian tarkkaan, eikä enää suoriudu yleisestä tapauksesta toivotulla tavalla. Ylisovitusta korjaamaan on kehitetty metodeja kuten esimerkiksi neuroniyksikköjen pudotus (dropout) jossa opetusvaiheessa yksittäisiä neuroneita poistetaan käytöstä, joka estää yksittäisiä neuroneita naapureineen erikoistumatta tiettyyn datan ominaisuuteen liian tarkasti.

  Neuroverkkojen opetuksessa olennaisessa osassa on myös verkon alkuperäisten painotuksien valitseminen sopivalla tavalla.

  \subsection{Hintafunktio}
    Olkoon $y(x)$ jokin funktio jota neuroverkko approksimoi. Hintafunktion avulla voidaan laskea kuinka paljon neuroverkon approksimaatio eroaa todellisesta funktiosta. Usein käytetään neliöllistä virhefunktiota:

    \begin{equation}
      \label{eq:cost-function}
      C = 1/2n \Sigma || y(x) - a ||^2
    \end{equation}
    jossa $a$ on taulukko neuroverkon saamia arvoja.

  \subsection{(stochastic?) Gradient descent (suomeksi)}
  batch gradient descent
  Ei välttämättä haluta ajaa takaisinvirtausalgoritmiä uudestaan jokaiselle harjoitus"materiaalin" yksikölle. Voidaan ottaa niistä vain jotain keskiarvoja tms eli batching skjfdghsg


  Hintafunktio vs takaisinvirtausalgoritmi vs gradient descent

  \subsection{Takaisinvirtausalgoritmi}
  Takaisinvirtausalgoritmi (back-propagation)

  \section{Konvolutionaalisten neuroverkkojen rakenne}
  Konvolutionaalisten tasojen ero täysin yhdistettyihin neuroverkkotasoihin. Konvolutionaaliset käsittelevät vain osaa kuvasta kerrallaan konvoluutioiden avulla.
  ReLu
  \subsection{Konvoluution matemaattinen selitys}
  \subsection{konvoluutiokerrokset}
  \subsection{pooling}
  kasautumiskerrokset?
  \subsection{flattening}
  Konvoluutiokerrosten yhdistäminen täysin yhdistettyjen kerrosten kanssa
  \subsection{full connection}

  \section{Konvolutionaalisten neuroverkkojen sovellukset}
  \subsection{Kuvien luokittelu}











  % ------------------------------ References ------------------------------
  %
  % bibtex is used to generate the bibliography. The babplain style
  % will generate numeric references (e.g. [1]) appropriate for theoretical
  % computer science. If you need alphanumeric references (e.g [Tur90]), use
  %
  % \bibliographystyle{babalpha-lf}
  %
  % instead.

  \nocite{*}
  \bibliographystyle{babplain-lf}
  \bibliography{references}


  % --- Appendices ---

  % uncomment the following

  % \newpage
  % \appendix
  %
  % \section{Esimerkkiliite}

  \end{document}
