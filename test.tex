\documentclass[11pt]{article}
  \renewcommand{\baselinestretch}{1.05}
  \usepackage{amsmath,amsthm,verbatim,amssymb,amsfonts,amscd,graphicx}
  \usepackage{graphics}
  \usepackage{url}
  \usepackage[utf8]{inputenc}
  \topmargin0.0cm
  \headheight0.0cm
  \headsep0.0cm
  \oddsidemargin0.0cm
  \textheight23.0cm
  \textwidth16.5cm
  \footskip1.0cm
  \theoremstyle{plain}
  \newtheorem{theorem}{Theorem}
  \newtheorem{corollary}{Corollary}
  \newtheorem{lemma}{Lemma}
  \newtheorem{proposition}{Proposition}
  \newtheorem*{surfacecor}{Corollary 1}
  \newtheorem{conjecture}{Conjecture}
  \newtheorem{question}{Question}
  \theoremstyle{definition}
  \newtheorem{definition}{Definition}


   \begin{document}

  \title{Konvolutionaariset neuroverkot}
  \author{Teemu Sarapisto}
  \maketitle

  \section{Introduction}


  Neuroverkot yleistyneet blablabla. Perustuvat joltain osin luonnollisiin neuroverkkoihin. herp \cite{Goodfellow-et-al-2016} derp


  \section{neuroverkkojen toiminta ylipäätään}
  \subsection{yksi neuroni / perseptroni}
  \subsection{monta neuronia}
  \subsection{suuremmat neuroverkot}
  \subsection{aktivaatiofunktio}
  \subsection{backpropagation}

  \subsection{Structure}
   The  line `` (backslash) begin \{ document \} "  introduces the body of the paper.  Any ``begin" command must get paired with an ``end" command; look at the last line in this template

   I can mention that it was in Section~\ref{section:mathmode}. Labels work for definitions, theorems, questions, sections, diagrams, and equations, among others.

  \subsection{Math Mode}\label{section:mathmode}
  % If you're wondering about the ``\label" above, it will be explained below.  Note that this line of text which follows the percent sign doesn't show up in the pdf.  This is a good way to leave notes for yourself on a work in progress.
   \textit{italic} or \textbf{boldface}),  $x+ y=7$ and the program takes care of spacing.  It's also easy to write Greek letters ($\alpha$, $\Sigma$), exponents ($2^{x+y}$), and subscripts ($x_1$)



   \bibliographystyle{ieeetr}
   \bibliography{test}

  \end{document}